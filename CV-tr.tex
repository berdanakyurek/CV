% !TeX spellcheck = en_US
\documentclass[
    10pt,
    A4,
    english,
    draft = false,
    twoside = false,
]{article}

\usepackage{curriculum-vitae}

\begin{document}
	%	Basic information
	\setname{Berdan}{Akyürek}
	\setaddress{Ankara / Türkiye}
	\setmobile{(+90) 544 798 58 41}
	\setmail{berdanakyurek17@gmail.com}
	\setlinkedin{https://www.linkedin.com/in/berdan-akyurek/}
	\setgithub{https://github.com/berdanakyurek}
	% \setblog{https://duck2345.com/}

	%---------------------------------------------------------------------------------------
	%	Title + Contact
	%----------------------------------------------------------------------------------------
	\cvtitle{Bilgisayar Mühendisi}
	%---------------------------------------------------------------------------------------
	%---------------------------------------------------------------------------------------
	%	Current Position
	%----------------------------------------------------------------------------------------
%	\cvSection{Current Position}
	%---------------------------------------------------------------------------------------
	%	Education
	%----------------------------------------------------------------------------------------
	\cvSection{EĞİTİM}
	\CVBlockWithTime{BİLKENT ÜNİVERSİTESİ}{Eylül 2016 - Şubat 2023}
	{Bilgisayar Mühendisliği, Lisans}{Ankara, Türkiye}
    {
      \CVBlockSubWithMidTitle{Pigeon's Map}{Bitirme Projesi}
      {Java ile geliştirilen, postacılar için rota planlama mobil uygulaması.}
    }
	\CVBlockWithTime{GÖLBAŞI ANADOLU LİSESİ}{Eylül 2014 - Haziran 2016}
	{Lise}{Ankara, Türkiye}{}
	%---------------------------------------------------------------------------------------
	%	Experience
	%----------------------------------------------------------------------------------------
	\cvSection{DENEYİM}
	\CVBlockWithTime{TEUS TEKNOLOJİ}{Mayıs 2024 - Devam Ediyor}{Full Stack Developer}{Ankara, Türkiye}
    {
      \CVBlockSub{SAĞLIK BAKANLIĞI MERKEZİ HASTA INDEX SİSTEMİ (MHIS)}
      {React ve .Net Core ile geliştirilen, tüm Sağlık Bakanlığı uygulamalarının entegre olacağı, hasta verilerini tutan merkezi bir platform.}
      \CVBlockSub{FINAPEX}
      {React ve .Net Core ile geliştirilen, fiş, fatura gibi mali dokümanların OCR ile okunarak  ve ilgili alanlarının ayrıştırılarak işlendiği bir mali müşavirlik yazılımı.}
    }
    \CVBlockWithTime{VENTURA YAZILIM}{Eylül 2022 - Aralık 2023}{Full Stack Developer}{Ankara, Türkiye}
    {
      \CVBlockSub{SAĞLIK BAKANLIĞI LABORATUVAR BİLGİ YÖNETİM SİSTEMİ (LBYS)}
      {Ulusal çapta kullanılmakta olan LBYS projesinin React ve .Net Core kullanılarak geliştirilmesi.}
      \CVBlockSub{ORIGO-HIS}
      {React ve .Net Core kullanılarak geliştirilen Hastane Bilgi Yönetim Sistemi (HBYS) projesi.}
    }
    \CVBlockWithTime{BK MOBİL}{Ağustos 2022 - Eylül 2022}{Stajyer}{Ankara, Türkiye}
    {
      \CVBlockSub{METODBOX}
      {Uğur Okulları, Bahçeşehir Koleji gibi okullarda kullanılan yapay zeka destekli uzaktan eğitim platformu.}
    }
    \CVBlockWithTime{NETCAD}{Ağustos 2020 - Eylül 2020}{Stajyer}{Ankara, Türkiye}
    {}
	%---------------------------------------------------------------------------------------
	%	Skills
	%----------------------------------------------------------------------------------------
	\cvSection{YETENEKLER}
	\tab \begin{tabular}{r p{0.7\textwidth}}
      \texttt{\large PROGRAMLAMA DİLLERİ} & \textbf{Deneyimli:} C\# \cvContactSep JavaScript \cvContactSep TypeScript \hfill \textbf{Aşina:} Python \cvContactSep C++ \cvContactSep C \cvContactSep Java \cvContactSep ELisp \\
      \texttt{\large FRAMEWORK \& TOOL'LAR} & .NET Core 8 \cvContactSep Entity Framework \cvContactSep React \cvContactSep Git \cvContactSep Linux \cvContactSep Emacs \\
      \texttt{\large DİLLER} & \textbf{Ana Dil:} Türkçe \hfill \textbf{İleri:} İngilizce (Bilkent COPE 90.25/150.00) \\
	\end{tabular}\\~\\
	% ---------------------------------------------------------------------------------------

\end{document}
