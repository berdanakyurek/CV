%-------------------------
% Resume in Latex
% Author : Sourabh Bajaj
% License : MIT
%------------------------

\documentclass[a4paper,11pt]{article}

\usepackage{latexsym}
\usepackage[empty]{fullpage}
\usepackage{titlesec}
\usepackage{marvosym}
\usepackage[usenames,dvipsnames]{color}
\usepackage{verbatim}
\usepackage{enumitem}
\usepackage[hidelinks]{hyperref}
\usepackage{fancyhdr}
\usepackage[T1]{fontenc}
\usepackage[utf8]{inputenc}
\usepackage[turkish]{babel} 
\usepackage{tabularx}
\usepackage{ragged2e}
\input{glyphtounicode}

\pagestyle{fancy}
\fancyhf{} % clear all header and footer fields
\fancyfoot{}
\renewcommand{\headrulewidth}{0pt}
\renewcommand{\footrulewidth}{0pt}

% Adjust margins
\addtolength{\oddsidemargin}{-0.5in}
\addtolength{\evensidemargin}{-0.5in}
\addtolength{\textwidth}{1in}
\addtolength{\topmargin}{-.5in}
\addtolength{\textheight}{1.0in}

\urlstyle{same}

\raggedbottom
\raggedright
\setlength{\tabcolsep}{0in}

% Sections formatting
\titleformat{\section}{
  \vspace{-4pt}\scshape\raggedright\large
}{}{0em}{}[\color{black}\titlerule \vspace{-5pt}]

% Ensure that generate pdf is machine readable/ATS parsable
\pdfgentounicode=1

%-------------------------
% Custom commands
\newcommand{\resumeItem}[2]{
  \item\small{
    \textbf{#1}{: #2 \vspace{-2pt}}
  }
}

\newcommand{\resumeItemShort}[1]{%
  \item\small{#1\vspace{-2pt}}%
}



% Just in case someone needs a heading that does not need to be in a list
\newcommand{\resumeHeading}[4]{
    \begin{tabular*}{0.99\textwidth}[t]{l@{\extracolsep{\fill}}r}
      \textbf{#1} & #2 \\
      \textit{\small#3} & \textit{\small #4} \\
    \end{tabular*}\vspace{-5pt}
}

\newcommand{\resumeSubheading}[4]{
  \vspace{-1pt}\item
    \begin{tabular*}{0.97\textwidth}[t]{l@{\extracolsep{\fill}}r}
      \textbf{#1} & #2 \\
      \textit{\small#3} & \textit{\small #4} \\
    \end{tabular*}\vspace{-5pt}
}

\newcommand{\resumeSubheadingShort}[2]{
  \vspace{-1pt}\item
    \begin{tabular*}{0.97\textwidth}[t]{l@{\extracolsep{\fill}}r}
      \textbf{#1} & #2 \\
    \end{tabular*}\vspace{-5pt}
}

\newcommand{\resumeSubSubheading}[2]{
    \begin{tabular*}{0.97\textwidth}{l@{\extracolsep{\fill}}r}
      \textit{\small#1} & \textit{\small #2} \\
    \end{tabular*}\vspace{-5pt}
}

\newcommand{\resumeSubItem}[2]{\resumeItem{#1}{#2}\vspace{-4pt}}

\renewcommand{\labelitemii}{$\circ$}

\newcommand{\resumeSubHeadingListStart}{\begin{itemize}[leftmargin=*]}
\newcommand{\resumeSubHeadingListEnd}{\end{itemize}}
\newcommand{\resumeItemListStart}{\begin{itemize}}
\newcommand{\resumeItemListEnd}{\end{itemize}\vspace{-5pt}}

%-------------------------------------------
%%%%%%  CV STARTS HERE  %%%%%%%%%%%%%%%%%%%%%%%%%%%%


\begin{document}

%----------HEADING-----------------
\begin{tabular*}{\textwidth}{l@{\extracolsep{\fill}}r}
  \textbf{\Large Berdan Akyürek} &  \href{mailto:berdanakyurek17@gmail.com}{berdanakyurek17@gmail.com}\\
  Ankara/Türkiye & \href{tel:+905447985841}{(+90) 544 798 58 41}\\
  \href{https://github.com/berdanakyurek}{https://github.com/berdanakyurek} & \href{https://www.linkedin.com/in/berdan-akyurek/}{LinkedIn/berdan-akyurek} \\
\end{tabular*}


%-----------EDUCATION-----------------

\section{ÖZET}
\justifying
Bilkent Üniversitesi Bilgisayar Mühendisliği mezunu, sağlık alanında deneyimli full-stack yazılım geliştiriciyim. Modern .NET ve React teknolojileriyle ölçeklenebilir çözümler geliştiriyorum. Mikroservisler ve yazılım mimarisi konularında uygulamalı tecrübeye sahibim.

%-----------EDUCATION-----------------
\section{EĞİTİM}
  \resumeSubHeadingListStart
    \resumeSubheadingShort
    {Lisans, Bilgisayar Mühendisliği, Bilkent Üniversitesi ( Eylül 2016 -- Şubat 2023)}{}
  \resumeSubHeadingListEnd


%-----------EXPERIENCE-----------------
\section{DENEYİM}
  \resumeSubHeadingListStart
  \resumeSubheading
  {TC Sağlık Bakanlığı}{Ankara, Türkiye}
  {Full Stack Developer (TEUS Technoloji aracılığıyla)}{Temmuz 2024 - Devam Ediyor}
  \resumeItemListStart
  \resumeItem{Merkezi Hasta İndeks Sistemi}{Hasta verilerinin merkezi yönetimi için tasarlanan Master Patient Index uygulamasının arayüzü ve çeşitli servislerinin geliştirilmesinde rol aldım (.NET Core 8, React).}
  \resumeItem{Ortak Giriş Noktası}{Sağlık Bakanlığı projeleri için merkezi oturum açma uygulamasının bakım çalışmalarına dahil oldum (.NET Core 3, React).}
  \resumeItemListEnd

  \resumeSubheading
  {TEUS Teknoloji}{Ankara, Türkiye}
  {Full Stack Developer}{Mayıs 2024 - Devam Ediyor}
  \resumeItemListStart
  \resumeItem{MyRoboTax}{Almanya’da yaygın olarak kullanılan DATEV sistemine entegre, sıfırdan geliştirilmiş bir mali müşavir otomasyon uygulamasını tasarlayıp geliştirdim (.NET Core 9, React).}
  \resumeItem{Notification Gateway}{Çeşitli sistemler için otomatik SMS ve e-posta süreçlerini yöneten web uygulamasının SMS modülünün ön ve arka yüz geliştirmesine katkı sağladım (.NET Core 8, React).}
  \resumeItemListEnd

  \resumeSubheading
  {Ventura Yazılım}{Ankara, Türkiye}
  {Full Stack Developer}{Eylül 2022 - Aralık 2023}
  \resumeItemListStart
  \resumeItem{Sağlık Bakanlığı Laboratuvar Bilgi Yönetim Sistemi (LBYS)}{Devlet laboratuvarlarında kullanılmakta olan LBYS projesinin modern teknolojilerle yeniden yazımına aktif olarak katkıda bulundum (.NET Core 8, React).}
  \resumeItem{ORIGO-HIS}{Yurt dışındaki hastanelerde kullanılmak üzere geliştirilen HIS projesinde çeşitli arayüzler tasarladım ve endpoint’ler geliştirdim (.NET Core 8, React).}
  \resumeItemListEnd

  \resumeSubheading   
  {BK Mobil}{Ankara, Turkiye}
  {Stajyer}{Ağustos 2022 - Eylül 2022}
  \resumeItemListStart
  \resumeItem{Metodbox}{Çevrim içi eğitim platformu için öğrenci kulüpleri modülünün ön yüzünü geliştirdim (JavaScript).}
  \resumeItemListEnd

  \resumeSubheading   
  {Netcad}{Ankara, Turkey}
  {Stajyer}{August 2020 - September 2020}
  \resumeItemListStart
  \resumeItem{Çizim Yazılımı PoC}{Çok katmanlı, görünürlükleri bağımsız olarak açılıp kapatılabilen nondestructive düzenleme destekli MS Paint benzeri bir çizim uygulamasının tasarımını ve PoC geliştirmesini yaptım (C\#).}
  \resumeItemListEnd
  
  \resumeSubHeadingListEnd


% --------PROGRAMMING SKILLS------------
\section{BECERİLER}
\resumeSubHeadingListStart
  \resumeSubheadingShort {Programlama Dilleri}{}
  \resumeItemListStart
    \resumeItem{Tecrübeli}{C\#, Javascript, Typescript}
    \resumeItem{Aşina}{Python, C++, Java, Elisp}
  \resumeItemListEnd
  \resumeSubheadingShort {Framework ve Tool'lar}{}
  \resumeItemListStart
    \resumeItemShort {.NET Core 8, Entity Framework, PostgreSQL, React, Microservisler, Docker, Redis, Kafka, Git, Linux, Emacs}
  \resumeItemListEnd
  \resumeSubheadingShort {Diller}{}
  \resumeItemListStart
    \resumeItem{Ana Dil}{Türkçe}
    \resumeItem{İleri}{İngilizce (YDS: 82.5/100)}
  \resumeItemListEnd
  \resumeSubHeadingListEnd


% --------ACHIEVEMENTS------------
\section{BAŞARILAR}
\resumeSubHeadingListStart
\resumeSubheading
  {ALES – Akademik Personel ve Lisansüstü Eğitimi Giriş Sınavı}{}
  {139,565 katılımcı arasında 395. sırada (ilk $\sim$\%0.3\ içerisinde) yer aldım.}{Aralık 2024}
\resumeSubHeadingListEnd


\end{document}
